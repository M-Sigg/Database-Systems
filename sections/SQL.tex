\begin{itemize}
    \item SQL is based on \textbf{multisets} (or bags) rather than sets. In a multiset an element may occur multiple times.
    \begin{itemize}
        \item Reasons for not automatically removing duplicates:
        \begin{itemize}
            \item expensive to remove duplicates (sorting)
            \item for aggregation duplicates are relevant
        \end{itemize}

    \end{itemize}
    \item We write \{...\} for a set and \{\{...\}\} for a bag
\end{itemize}

\begin{table}[H]
\centering
\begin{tabular}{|c|c|c|}
    \specialrule{0.01em}{0em}{0em} 
     \textbf{SQL} & \textbf{Relational Algebra} & \textbf{Domain Relational Calculus}  \\
    \specialrule{0.01em}{0em}{0em} 
     table & relation & predicate \\
    \specialrule{0.005em}{0em}{0em} % Fine line with specified thickness and spacing
     column & attribute & argument\\
    \specialrule{0.005em}{0em}{0em}  % Fine line with specified thickness and spacing
    row & tuple & - \\
    \specialrule{0.005em}{0em}{0em}  % Fine line with specified thickness and spacing
    query & RA expression & formula \\
     \specialrule{0.01em}{0em}{0em} 
\end{tabular}
\end{table}

\subsection{Data Definition Language}

\subsubsection{Domain Types}
Domain Types in SQL:
\begin{itemize}
    \item CHAR(n): fixed length character string
    \item VARCHAR(n): character string with maximum length $n$
    \item INTEGER
    \item SMALLINT
    \item NUMERIC(p,d): Fixed point number, user-specified precision of $p$ digits, with $n$ digits to the right of decimal point
    \item REAL, DOUBLE PRECISION: Floating point and double-precision floating point numbers
    \item FLOAT(n): Floating point number, with user-specified precision of at least $n$ digits
\end{itemize}
\subsubsection{Create, dropping and altering tables}
\textbf{\underline{Create Table Construct}}
\bigskip

Example: 
\begin{lstlisting}[caption= Create Table Example]
    CREATE TABLE branch (
        BranchName CHAR(15),
        BranchCity CHAR(30),
        Assets INTEGER )
\end{lstlisting}

\textbf{\underline{Drop and Alter Table Constructs}}
\bigskip
\begin{itemize}
    \item The \textbf{drop table} command deletes the whole table from the database
    \item The \textbf{alter table} command is used to add columns to an existing table
    \begin{itemize}
        \item[] \textbf{alter table} r \textbf{add} $A$ $D$ 
    \end{itemize}
\end{itemize}


\subsubsection{Integrity constrains}

\begin{itemize}
    \item Integrity constraints guard against damage to the database
    \item Formally, an integrity constraint is a closed first order formula that must be true, i.e., that each database instance must satisfy.
    \begin{itemize}
        \item Example: $\forall B(account(\_,\_,B) \Rightarrow B<10M)$
    \end{itemize}
\end{itemize}

Integrity constraints on single relations:
\begin{itemize}
    \item \textbf{Domain Constraints}
    \begin{itemize}
        \item Domain constraints are the most elementary form of integrity constraints. They check values inserted in the database, and they check queries to ensure that the comparisons make sense.
        \item New domains can be created from existing data types
        \item Example:
        \begin{itemize}
            \item[] \begin{lstlisting}
                CREATE DOMAIN Dollars INTEGER
                CREATE DOMAIN Pounds INTEGER
            \end{lstlisting}
        \end{itemize}
        \item We cannot assign or compare a value of type Dollar to a value of type Pounds. But we can convert values: 
        \begin{itemize}
            \item[] \begin{lstlisting}
                CAST(r.Amnt/1.5 AS Pounds)
            \end{lstlisting}
        \end{itemize}
    \end{itemize}
    \item \textbf{not null}
    \begin{itemize}
        \item[] \begin{lstlisting}
            BranchName VARCHAR(15) NOT NULL
            CREATE DOMAIN Dollars INTEGER NOT NULL
        \end{lstlisting}
    \end{itemize}
    \item \textbf{primary key}
    \begin{itemize}
        \item A primary key ensures that an attribute value is not null and unique across all rows of a table. Therefore it can be used to identify a unique row in a table, and is used for query optimization
        \item Within the Create Table construct include \begin{lstlisting}
            PRIMARY KEY (CustName)
        \end{lstlisting}
    \end{itemize}
    \item \textbf{unique}
    \begin{itemize}
        \item The unique specification states that the attributes $A_1,A_2,...,A_m$ form a candidate key.
        \item Candidate keys are permitted to be null 
    \end{itemize}
    \item \textbf{check}($P$), where $P$ is a predicate over a single relation 
    \begin{itemize}
        \item Within the Create Table Construct include: \begin{lstlisting}
            CHECK (Assets >= 0)
        \end{lstlisting}
    \end{itemize}
\end{itemize}

Integrity constraints over multiple relations (Referential Integrity):
\begin{itemize}
    \item \textbf{foreign key}
    \begin{itemize}
        \item Within the Create Table Construct include:
        \begin{lstlisting}
            FOREIGN KEY (BranchN) REFERENCES branch
        \end{lstlisting}
    \end{itemize}
    \item \textbf{check}($P$), where $P$ is a predicate over multiple relations
    \item \textbf{assertion}
    \begin{itemize}
        \item An assertion is a predicate expressing a condition that the database must satisfy.
        \item An assertion in SQL takes the form
        \begin{itemize}
            \item[] \textbf{create assertion} <assertion-name> \textbf{check} <predicate>
        \end{itemize}
    \end{itemize}
\end{itemize}
\subsection{Expressions and Predicates}
\textbf{\underline{Expressions}}
\bigskip

\begin{itemize}
    \item Numeric functions: $+,-,*,/, abs(e), ceil(e), tan(e), log(e), round(e), sign(e), mod(e,f),...$
    \item Character functions:
    \begin{itemize}
        \item concatenation: 'abc' || 'de' = 'abcde'
        \item \textbf{position}('48' \textbf{in} 'another 48 hours') = 9
        \item \textbf{substring}('anothe' \textbf{from} 1 \textbf{for} 3) = 'ano'
        \item \textbf{upper}(e)
        \item \textbf{trim}(\textbf{leading} ' ' \textbf{from} ' test') = 'test
    \end{itemize}
    \item date and time functions:
    \begin{itemize}
        \item \textbf{current\_date}, \textbf{current\_time}
        \item \textbf{current\_timestamp}
        \item \textbf{current\_date} + \textbf{interval} '1' \textbf{day}
        \item \textbf{current\_date} - \textbf{date} '1990/3/18'
        \item \textbf{interval} '6' \textbf{day} - \textbf{interval} 'a' \textbf{day} = \textbf{interval} '5' \textbf{day}
     \end{itemize}
     \item type conversions:
     \begin{itemize}
         \item \textbf{cast}('48' \textbf{as integer}) = 48
         \item often "natural" type conversions are done implicitly
     \end{itemize}
     \item conditional statement:
     \begin{lstlisting}
         case
         when cond1 then result1
         ...
         when condN then resultN
         else resultX
         end
     \end{lstlisting}
     \item \textbf{coalesce}(val1,..., valN)
     \begin{itemize}
         \item returns the first value that is not null (and null if all values are equal to null)
     \end{itemize}
     \item  \textbf{nullif}(e1,e2)
     \begin{itemize}
         \item returns null if e1 and e2 are identical; useful if missing information is not represented with null; \textbf{nullif}(cost,9999)
     \end{itemize}
\end{itemize}

\textbf{\underline{Predicates}}
\bigskip

\begin{itemize}
    \item Predicates evaluate to true, false or unknown
    \item boolean connectives: \textbf{and}, \textbf{or}, \textbf{not}
    \item $=,<>, <,>,<=,>=$
    \item e \textbf{[not] between} e1 \textbf{and} e2
    \item e \textbf{is [not] null}
    \item e \textbf{in} (e1,...,eN)
    \item e1 \textbf{like} e2
    \begin{itemize}
        \item example:
        \begin{lstlisting}
            Name like '_ross%'
        \end{lstlisting}
        \item wildcards: \% 0-n characters, \_ 1 character
    \end{itemize}
\end{itemize}



\subsection{Table Expressions, Query Specifications, Query Expressions}
\textbf{\underline{Structure of SQL Queries:}}
\bigskip

\begin{figure}[H]
\centering
\includegraphics[width=0.6\textwidth]{images/Screenshot 2024-05-04 at 16.56.14.jpg}
\caption{SQL Structure}
\end{figure}

The result of an SQL query is a (virtual) table.

\subsubsection{The from Clause}

\begin{itemize}
    \item The \textbf{from} clause lists the table involved in the query
    \begin{itemize}
        \item Corresponds to the Cartesian product operation of the relational algebra (if multiple tables listed in from clause)
    \end{itemize}
    \item Many different joins supported
    \begin{itemize}
        \item \textbf{CROSS JOIN} = Cartesian product (implicitly also given without keyword)
        \item \textbf{Theta Join}: \quad \textbf{FROM} t1 \textbf{JOIN} t2 \textbf{ON} t1.a < t2.b
        \item \textbf{Left Outer Join}: \quad \textbf{FROM} t1 \textbf{LEFT OUTER JOIN} t2 \textbf{ON} t1.a = t2.b
        \item \textbf{Natural Join}:  \quad \textbf{FROM} t1 \textbf{NATURAL INNER JOIN} t2 
        \item \textbf{Limited Natural Join} (not all pair-wise equal columns are used for the natural join; only the ones specified in the using clause):  \quad \textbf{FROM} t1 \textbf{NATURAL INNER JOIN} t2 \textbf{USING} (name)
    \end{itemize}
\end{itemize}

\subsubsection{The where Clause}

\begin{itemize}
    \item The \textbf{where} clause specifies conditions that the result tuples must satisfy
    \item Example:
    \begin{lstlisting}
        FROM loan
        WHERE BranchName = 'Perryridge' 
            AND Amount > 1200
    \end{lstlisting}
    \item logical connectives: and, or, not
\end{itemize}

\subsubsection{The group Clause}
\begin{itemize}
    \item The \textbf{group} clause takes the table produced by the where clause and returns a grouped table
    \item Example:
    \begin{lstlisting}
        FROM account
        GROUP BY BranchName
    \end{lstlisting}
    \item The following statements work on the groups (and not on individual tuples)
\end{itemize}

\subsubsection{The having Clause}
\begin{itemize}
    \item The \textbf{having} clause takes a grouped table as input and returns a grouped table
    \item The having condition is applied to each group. Only groups that satisfy it are returned
    \item Example:
    \begin{lstlisting}
        FROM account
        GROUP BY BranchName
        HAVING COUNT(AccNr) > 1
    \end{lstlisting}
\end{itemize}

\subsubsection{The select Clause}
\begin{itemize}
    \item The \textbf{select} clause lists the columns that shall be in the result of a query
    \begin{itemize}
        \item Corresponds to the projection operation of RA
    \end{itemize}
    \item To force the elimination of duplicates, insert the keyword \textbf{distinct} after select
    \item In the select clause aggregate functions can be used:
    \begin{itemize}
        \item avg, min, max, sum, count
    \end{itemize}
\end{itemize}

\subsubsection{Derived Tables}
\begin{itemize}
    \item SQL allows a query expression to be used in the \textbf{from} clause
    \item Example:
    \begin{lstlisting}
        SELECT BranchName, AvgBalance
        FROM ( SELECT BranchName AVG(Balance)
               FROM account
               GROUP BY BranchName
             ) AS branchAvg(BranchName, AvgBalance)
        WHERE AvgBalance > 1200
    \end{lstlisting}
\end{itemize}

\subsubsection{Query Expressions}
\begin{itemize}
    \item The set operations \textbf{union}, \textbf{intersect}, and \textbf{except} operate on tables and correspond to the relational algebra operations $\cup, \cap, -$
    \item Each automatically removes duplicates, to preserve duplicates add \textbf{all} after operator
\end{itemize}

\subsection{Subqueries, Duplicates, Null Values}

\subsubsection{Subqueries}

\begin{itemize}
    \item A \textbf{subquery} is a \textbf{query expression} that is nested within another query expression.
    \item Example:
    \begin{lstlisting}
        SELECT X FROM p WHERE X IN (SELECT Y FROM q)
    \end{lstlisting}
    \item[\textbf{!}] Semantics/evaluation: For each tuple of p evaluate the subquery and check if X is in the result table computed by the subquery
    \item The \textbf{exists} construct returns the value \textbf{true} if the argument subquery is nonempty
    \begin{itemize}
        \item[\textbf{!}] Use \textbf{exists} instead of \textbf{in, all, any, some}
    \end{itemize}
\end{itemize}

\subsubsection{Duplicates}

SQL duplicate semantics:\\

\quad \textbf{select} $A_1,A_2,...,A_n$ \

\quad \textbf{from} $r_1, r_2,...,r_m$ \

\quad \textbf{where} $P$\\

is equivalent to the \textit{multiset} version of the expression: \[
\pi_{A_1,A_2,...,A_n}(\sigma_P (r_1 \times r_2 \times ... \times r_m))
\]

\subsubsection{Null Values}
\begin{itemize}
    \item The  predicate \textbf{is null} can be used to check for null values.
    \item Any comparison with \textit{null} returns unknown
    \begin{itemize}
        \item Example: $5<null$ or $null<>null$ or $null=null$ 
    \end{itemize}
    \item All aggregate operations, expect \textbf{count(*)}, ignore tuples with null values on the aggregated columns. 
    \begin{itemize}
        \item Arithmetic operations like $7+ null = null$, thus they do not ignore null values
    \end{itemize}
\end{itemize}

\subsubsection{Ordering}

\begin{itemize}
    \item \textbf{ORDER BY} CustName => orders the result table alphabetically (ascending = default)
    \item \textbf{ORDER BY} CustName \textbf{DESC} => in descending order
\end{itemize}

\subsection{Modification of the Database}

\subsubsection{Insertions}
\begin{lstlisting}
    INSERT INTO account
        VALUES ('A-9732', 'Perryridge', 1200)
\end{lstlisting}

\subsubsection{Deletions}
\begin{lstlisting}
    DELETE FROM account
    WHERE BranchName = 'Perryridge'
\end{lstlisting}
\subsubsection{Updates}
\begin{lstlisting}
    UPDATE account
    SET Balance = Balance * 1.06
    WHERE Balance > 10000

    UPDATE account 
    SET Balance = Balance * 1.05
    WHERE Balance <= 10000
\end{lstlisting}
Here order matters! $\Rightarrow$ Better use the \textbf{case} statement 
\begin{lstlisting}
    UPDATE account
    SET Balance = CASE 
                    WHEN Balance <= 10000
                    THEN Balance * 1.05
                    ELSE Balance * 1.06
                  END
\end{lstlisting}   

\subsection{Views and Recursion}

\subsubsection{Purpose of views}

\subsubsection{Creation and use of views}

\subsubsection{Handling views in the DBMS}

\subsubsection{Temporary Views}


\subsection{User Defined Functions}