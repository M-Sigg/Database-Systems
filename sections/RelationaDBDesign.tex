\subsection{Relational Database Design Guidelines}

\subsubsection{Goals of Relational Database Design}
The goal of relational database design is to find a good collection of relation schemas. The main problem is to find a good grouping of the attributes into relation schemas.\\
We have a good collection of relation schemas if we:
\begin{itemize}
    \item Ensure a simple semantics of tuples and attributes
    \item Avoid redundant data
    \item Avoid update anomalies
    \item Avoid null values as much as possible
    \item Ensure that exactly the original data is recorded and (natural) joins do not generate spurious tuples
\end{itemize}\

\textbf{Guideline 1:} Each tuple in a relation should only represent one entity or relationship instance \

\textbf{Guideline 2:} Design a schema that does not suffer from insertion, deletion and update anomalies.\

\textbf{Guideline 3:} Relations should be designed such that their tuples will have as few NULL values as possible; attributes that are NULL shall be placed in separate relations (along with the primary key)\

\textbf{Guideline 4:} Relations should be designed such that no spurious (i.e., wrong) tuples are generated if we do a natural join of the relations

\subsection{Functional Dependencies}
Reminder Key Constraints: \ref{keyConstraints}


\subsubsection{Definition}

Functional dependencies (FDs) are used to specify \textit{formal measures} of the goodness of relation designs. FDs and keys are used to define \textbf{normal forms} for relations. Functional Dependencies are \textbf{constraints} that are derived from the \textit{meaning} and \textit{interrelationships} of the attributes.\\

A set of attributes X \textit{functionally determines} a set of attributes Y if the value of X determines a unique value for Y. \[X \rightarrow Y\]

A functional dependency $X\rightarrow Y$ is \textbf{trivial} iff $Y\subseteq X$: \quad $(\underset{X}{FN,LN} \rightarrow \underset{Y}{FN})$ \label{trivial}\\

\begin{itemize}
    \item A FD constraint must hold on \textit{every} relation instance r(R)
    \item If K is a candidate key of R, the K functionally determines all attributes in R (since we never have two distinct tuples with $t_1[K]=t_2[K]$ 
\end{itemize}



\subsubsection{Armstrong's inference rules}
\begin{itemize}
    \item Given a set of FDs $F$, we can \textbf{infer} additional FDs that hold whenever the FDs in $F$ hold
    \item Armstrong's inference rules:
    \begin{itemize}
        \item Reflexivity: $Y\subseteq X \models X\rightarrow Y$
        \item Augmentation: $X \rightarrow Y \models XZ \rightarrow YZ$
        \item Transitivity: $X\rightarrow Y, Y \rightarrow Z \models X \rightarrow Z$
    \end{itemize}
    \item Armstrong's inference rules are \textbf{sound} and \textbf{complete}
    \item Additional inference rules that are useful:
    \begin{itemize}
        \item Decomposition: $X\rightarrow YZ \models X \rightarrow Y, X \rightarrow Z$
        \item Union: $X \rightarrow Y, X \rightarrow Z \models X \rightarrow YZ$
        \item Pseudotransitivity: $X \rightarrow Y, WY \rightarrow Z \models WX \rightarrow Z$
    \end{itemize}
\end{itemize}

\subsubsection{Closure and minimal cover}
\begin{itemize}
    \item The \textbf{closure} of a set t $F$ of FDs is the set $F^{+}$ of all FDs that can be inferred from $F$
    \item The \textbf{closure} of a set of attributes $X$ with respect to $F$ is the set $X^+$ of all attributes that are functionally determined by $X$
    \item Two sets of FDs $F$ and $G$ are \textbf{equivalent} if:
    \begin{itemize}
        \item Every FD in $F$ can be inferred from $G$, and
        \item Every FD in $G$ can be inferred from $F$
        \item Hence, $F$ and $G$ are equivalent if $F^+ = G^+$
    \end{itemize}
    \item Definition: $F$ \textbf{covers} $G$ if every FD in $G$ can be inferred from $F$ (i.e., if $G^+ \subseteq F^+$)
    \item A set of FDs is \textbf{minimal} if it satisfies the following conditions:
    \begin{enumerate}
        \item No pair of FDs has the same left-hand side 
        \item We cannot remove any dependency from $F$ and have a set of dependencies that is equivalent to $F$ 
        \item We cannot replace any dependencies $X\rightarrow A$ in $F$ with a dependency $Y\rightarrow A$, where $Y\subset X$ and still have a set of dependencies that is equivalent to $F$
    \end{enumerate}
\end{itemize}

\subsection{Normal Form}
\begin{itemize}
    \item \textbf{Normalization}: The process of decomposing bad relations by breaking up their attributes into smaller relations that satisfy the normal forms. 
    \item A normalized database consists of a good collection of relation schemas
    \item The database designers \textit{need not} normalize to the highest possible normal form
    \begin{itemize}
        \item usually they choose 3NF, BCNF or 4NF 
        \item controlled redundancy is OK/good
    \end{itemize}
\end{itemize}

\subsubsection{First Normal Form (1NF)}
Dissallows: \begin{itemize}
    \item composite attributes
    \item multivalued attributes 
    \item nested relations: attributes whose values for an \textit{individual tuple} are relations 
\end{itemize}
Remedy to get 1NF: Form new relations for each multivalued attribute or nested relation

\subsubsection{Second Normal Form (2NF)}
\begin{itemize}
    \item A relation schema R is in \textbf{second normal form (2NF)} iff each attribute not contained in a candidate key is not partially functional dependent on a candidate key of R. 
    \item An attribute is \textit{partially functional dependent} on a candidate key if it is functionally dependent on a proper subset of the candidate key
    \item Example: given candidate key SSN, PNum the FD $SSN \rightarrow EName$ would violate 2NF as EName is partially functional dependent on the candidate key (or fully funtional dependent on the subset SSN of the candidate key)
    \item \textbf{Remedy}: Decompose and set up a new relation for each partialy key with its dependent attributes. Keep a relation with the original key and any aatributes that are functionally dependent on it. 
\end{itemize}

\subsubsection{Third Normal Form (3NF)}

\begin{itemize}
    \item A relation schema R is in \textbf{third normal form (3NF)} iff for all $X\rightarrow A \in F^+$ at least \textbf{one} of the following holds:
    \begin{itemize}
        \item $X\rightarrow A$ is trivial \ref{trivial}
        \item X is a superkey for R
        \item A is contained in a candidate key of R
    \end{itemize}
    \item Intuition: "Each non-key attribute must describe the key, the whole key, and nothing but the key."
    \item A relation that is in 3NF is also in 2NF 
\end{itemize}

\subsubsection{Boyce-Codd Normal Form (BCNF)}
\begin{itemize}
    \item A relation schema R is in \textbf{Boyce-Codd Normal Form (BCNF)} iff for all $X\rightarrow A \in F^+$ at least one of the following holds:
    \begin{itemize}
        \item $X\rightarrow A$ is trivial
        \item X is a superkey of R  
    \end{itemize}
    \item Remedy to get BCNF: split relation
\end{itemize}

\subsection{Properties of Decomposition and Normalization Algorithm}
\begin{itemize}
    \item Relational database design by decomposition:
    \begin{itemize}
        \item Universal Relation Schema: A relation schema $R(A_1,A_2,,...,A_n)$ that includes all attributes of the database
        \item Decomposition: decompose the universal relation schema R into a set of relation schemas $D=R1,R2,...,Rm$ by using the functional dependencies
        \item Additional conditions:
        \begin{itemize}
            \item Each attribute in R will appear in at least one relation schema Ri in the decomposition so that no attributes are lost
            \item Have each individual relation Ri in the decomposition D in 3NF (or higher)
            \item \textbf{Lossless join decomposition}: ensures that the decomposition does not introduce wrong tuples when relations are joined together [must have]
            \item \textbf{Dependency preservation}: ensures that all functional dependency can be checked by considering individual relations Ri only [nice to have]
        \end{itemize}
    \end{itemize}
\end{itemize}

\subsubsection{Dependency Preservation}
\begin{itemize}
    \item Given a set of dependencies $F$ on $R$, the \textbf{projection} of $F$ on Ri, denoted by $F|Ri$ where $attr(Ri)$ is a subset of $attr(R)$, is the set of dependencies $X\rightarrow Y$ in $F^+$ such that the attributes in $X \cup Y$ are all contained in $attr(Ri)$
    \item Hence, the projection of $F$ on each relation schema $Ri$ in the decomposition $D$ is the set of functional dependencies in $F^+$, the closure of $F$. such that all their left- and right-hand side attributes are in $attr(Ri)$
    \item Dependency Preservation:
    \begin{itemize}
        \item A decomposition $D=R1,R2,...,Rm$ of R is \textbf{dependency-preserving} with respect to F if the union of the projections of F on each Ri in D is equivalent to F; that is \[
        (F|R1 \cup ...\cup F|Rm)^+ = F^+
        \]
    \end{itemize}
    \item It is always possible to find a dependency-preserving decomposition such that each relation is in 3NF
    \item It is not always possible to find a dependency-preserving decomposition such that each relation is in BCNF
\end{itemize}
\subsubsection{Lossless Join Decomposition}
\begin{itemize}
    \item A decomposition $D=R1,R2,...,Rm$ of R is a \textbf{loss less join decomposition} with respect to the set of dependencies $F$ on $R$ if, for every relation instance $r$ of $R$ that satisfies $F$, the following holds: \[
    \pi_{R1}(r) \bowtie ... \bowtie \pi_{attr(Rm)}(r)=r
    \] (join of the pieces gives the original relation
    \item R1 and R2 form a lossless join decomposition of R with respect to a set of functional dependencies F iff 
    \begin{itemize}
        \item $(R1 \cap R2) \rightarrow (R1-R2)$ is in $F^+$ or
        \item $(R1 \cap R2) \rightarrow (R2-R1)$ is in $F^+$
    \end{itemize}
    \item Example: Consider $R(A,B,C)$, $F=\{AB\rightarrow C, C \rightarrow B\}$, $R1(A,C), R2(B,C)$. Is $R1, R2$ a lossless decomposition of R?
    \begin{itemize}
        \item $(R1 \cap R2) \rightarrow (R1-R2) \Rightarrow C \rightarrow A$. This is not in $F^+$
        \item $(R1 \cap R2) \rightarrow (R2-R1) \Rightarrow C \rightarrow B$. This is in $F^+$ 
        \item R1, R2 is thus lossless a decomposition of R
    \end{itemize}
\end{itemize}

\subsubsection{BCNF Normalization Algorithm}
\begin{algorithm}
\caption{Decompose relation schema into BCNF}
\begin{algorithmic}[1] % The number [1] ensures lines are numbered
\State Set $D := \{R\}$
\While{\textit{a relation schema $Q$ in $D$ is not in BCNF}}
    \State find a functional dependency $X \rightarrow Y$ in $Q$ that violates BCNF;
    \State replace $Q$ in $D$ by two relation schemas $(Q - Y)$ and $(X \cup Y)$;
\EndWhile
\end{algorithmic}
\end{algorithm}
\quad \textit{Assumption: No null values are allowed for the join attributes}\\
The result is a lossless join decomposition. The resulting schemas do not necessarily preserve all dependencies.

\subsection{Multivalued Dependencies, 4NF}
\textbf{Definition:}
\begin{itemize}
    \item A \textbf{multivalued dependency (MVD)} $X \twoheadrightarrow Y$ on relation schema $R$, where $X$ and $Y$ are both subsets of $R$, specifies the following constraint on any relation instance $r$ of $R$:\\
    If two tuples $t_1$ and $t_2$ exist in $r$ such that $t_1[X]=t_2[X]$, then two tuples $t_3$ and $t_4$ should also exist in $r$ with the following properties, where we use $Z$ to denote $(R-(X\cup Y))$:
    \begin{itemize}
        \item $t_3[X]=t_4[X]=t_1[X]=t_2[X]$
        \item $t_3[Y]=t_4[Y]=t_1[Y]=t_2[Y]$
        \item $t_3[Z]=t_4[Z]=t_1[Z]=t_2[Z]$
    \end{itemize}
    \item A MVD $X \twoheadrightarrow Y$ for $R$ is called a \textbf{trivial MVD} if $Y \subseteq X$ or $X\cup Y = attr(R)$
\end{itemize}

\textbf{Inference Rules for Functional and Multivalued Dependencies}:
\begin{itemize}
    \item reflexivity FDs: $X\supseteq Y \models x \rightarrow Y$
    \item augmentation FDs: $X\rightarrow Y \models XZ \rightarrow YZ$
    \item transitivity FDs: $X\rightarrow Y, Y \rightarrow Z \models X \rightarrow Z$
    \item complementation: $X \twoheadrightarrow Y \models X \twoheadrightarrow (R-(X\cup Y))$
    \item augmentation MVDs: $X\twoheadrightarrow Y, W \supseteq Z \models WX \twoheadrightarrow YZ$
    \item transitivity MVDs: $X \twoheadrightarrow Y, Y \twoheadrightarrow Z \models X \twoheadrightarrow (Z-Y)$
    \item replication: $X\rightarrow Y \models X \twoheadrightarrow Y$
    \item coalescing: $X \twoheadrightarrow Y, \exists W(W\cap Y = \emptyset, W \rightarrow Z, Y \supseteq Z) \models X \rightarrow Z$ 
\end{itemize}


\subsubsection{Fourth Normal Form (4NF)}

\textbf{Definition:}
\begin{itemize}
    \item A relation schema $R$ with a set of functional and multivalued dependencies F is in \textbf{4NF} iff, for every multivalued dependency $X \twoheadrightarrow Y$ in $F^+$ at least one of the following holds:
    \begin{itemize}
        \item $X\twoheadrightarrow Y$ is trivial
        \item X is a superkey for $R$
    \end{itemize}
    \item If a relation is not in 4NF because of the MVD $X \twoheadrightarrow Y$ we decompose R into $R1(X\cup Y)$ and $R2(R-Y)$
    \item Such a decomposition is lossless
\end{itemize}

\subsection{Summary}
\begin{itemize}
    \item Relational database design goals: eliminate redundancy
    \item Main concept: functional dependencies
    \item Functional dependencies:
    \begin{itemize}
        \item Definition
        \item Armstrong's inference rules: reflexivity, augmentation, transitivity
        \item equivalence of sets of FDs
        \item minimal sets of FDs
    \end{itemize}
    \item Approach: Start with all attributes in a single relation and decompose it vertically until all functional dependencies are acceptable
    \item Normal forms based on candidate keys and FD
    \begin{itemize}
        \item 1NF, 2NF, 3NF, BCNF
    \end{itemize}
    \item BCNF normalization algorithm 
    \item Dependency Preservation
    \begin{itemize}
        \item always possible for 3NF; not always possible for BCNF
    \end{itemize}
    \item Lossless Join Decomposition
    \begin{itemize}
        \item always required
    \end{itemize}
    \item Multivalued Dependencies, 4NF
\end{itemize}